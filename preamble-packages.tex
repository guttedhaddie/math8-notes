\usepackage[utf8]{inputenc}

\usepackage{mathpazo}
\usepackage{amssymb,amsthm}
\usepackage[fleqn]{amsmath}
%\usepackage{bigints}
\usepackage[svgnames]{xcolor}
\usepackage{graphicx}
%\usepackage{animate}
%\usepackage{media9}
%\usepackage{manfnt}
%\usepackage{textcomp}
\usepackage{polynom}
%\usepackage[all]{xy}
\usepackage{hyperref}
\usepackage[breakable,theorems,skins]{tcolorbox}

\graphicspath{{asy/}}

\allowdisplaybreaks

\parindent0pt

%lengths
\unitlength 1cm
\textheight 22cm
\textwidth 17cm
\oddsidemargin -0.5cm
\evensidemargin -0.5cm
\topmargin -1.5cm
\topskip 0cm
\headheight 0.5cm
\headsep 1cm
\marginparwidth 1.2cm
\newlength\doubleind
\addtolength{\doubleind}{\leftmargini}
\addtolength{\doubleind}{\leftmarginii}

%MathNumbering
\makeatletter
\@addtoreset{equation}{section}
\renewcommand{\theequation}{\thesection.\@arabic\c@equation}
\newcommand\Nopagebreak{\@nobreaktrue\nopagebreak}
%\renewcommand{\theequation}{\@arabic\c@equation}
\makeatother

\newcommand\boldinline[1]{\paragraph{#1}}
\newcommand\boldsubsection[1]{\subsection*{#1}}
\newcommand\boldsubsubsection[1]{\subsubsection*{#1}}

\def\exstart{\hangindent\leftmargini\textup{1.}\hspace{\labelsep}}


%Theorems
\tcbset{
	defstyle/.style={enhanced, top=3pt, bottom=3pt, colframe=black, coltitle=black, arc=5pt, boxrule=1.5pt, left*=0pt, right*=0pt, theorem style=plain, terminator sign={.\ \ \ }, fonttitle=\bfseries\upshape, fontupper=\upshape, colback=blue!8!white, grow sidewards by=8pt, drop fuzzy shadow},
	exstyle/.style={enhanced, breakable, beforeafter skip balanced=10pt, coltitle=black, theorem style=plain, terminator sign={.\ \ \ }, fonttitle=\bfseries\upshape, fontupper=\upshape, blanker, borderline west={4pt}{-8pt}{orange!75!white}},
	thmstyle/.style={enhanced, top=3pt, bottom=3pt, colframe=black, coltitle=black, arc=5pt, boxrule=1.5pt,left*=0pt, right*=0pt, theorem style=plain, terminator sign={.\ \ \ }, fonttitle=\bfseries\upshape, fontupper=\slshape, colback=green!12!white, grow sidewards by=8pt, drop fuzzy shadow},
	exercisestyle/.style={enhanced, breakable, beforeafter skip balanced=10pt, coltitle=black, theorem style=plain, terminator sign={.\ \ \ }, fonttitle=\bfseries\upshape, fontupper=\upshape, blanker, borderline west={4pt}{-8pt}{purple!75!white}, title={Exercises\ \ \ }},
	proofstyle/.style={enhanced, breakable, beforeafter skip balanced=10pt, blanker, borderline west={4pt}{-8pt}{green!50!white}},
	asidestyle/.style={enhanced, breakable, beforeafter skip balanced=10pt, blanker, borderline west={4pt}{-8pt}{black!50!white}},
	exercisestyle2/.style={enhanced, breakable, beforeafter skip balanced=10pt, coltitle=black, theorem style=plain, terminator sign={.\ \ \ }, fonttitle=\bfseries\upshape, fontupper=\upshape, blanker, borderline west={4pt}{-8pt}{purple!75!white}, title={Exercises\ \thesubsection.\ \ \ }},
	exercisestyle4/.style={enhanced, breakable, beforeafter skip balanced=10pt, coltitle=black, theorem style=plain, terminator sign={.\ \ \ }, fonttitle=\bfseries\upshape, fontupper=\upshape, blanker, borderline west={4pt}{-8pt}{purple!75!white}, title={Exercises\ \thesection.\ \ \ }},
	exercisestyle3/.style={enhanced, breakable, beforeafter skip balanced=10pt, coltitle=black, theorem style=plain, terminator sign={.\ \ \ }, fonttitle=\bfseries\upshape, fontupper=\upshape, blanker, borderline west={4pt}{-8pt}{purple!75!white}, title={Exercise\quad}},
	}

\tcolorboxenvironment{proof}{breakable,proofstyle}


\newenvironment{proofpic}{%\topsep\relax
  \trivlist
  \item[\hskip\labelsep
        \itshape
    Proof.]\ignorespaces
}{\endtrivlist}


\newtcbtheorem[number within=section]{defn}{Definition}{defstyle}{defn}
\newtcbtheorem[use counter from=defn]{example}{Example}{exstyle}{ex}
\newtcbtheorem[use counter from=defn]{examples}{Examples}{exstyle}{ex}
\newtcbtheorem[use counter from=defn]{thm}{Theorem}{thmstyle}{thm}
\newtcbtheorem[use counter from=defn]{lemm}{Lemma}{thmstyle}{lemm}
\newtcbtheorem[use counter from=defn]{cor}{Corollary}{thmstyle}{cor}
\newtcbtheorem[use counter from=defn]{axiom}{Axiom}{defstyle}{axiom}
\newtcbtheorem[use counter from=defn]{axioms}{Axioms}{defstyle}{axioms}
\newtcbtheorem[use counter from=defn]{conj}{Conjecture}{thmstyle}{conj}

\newtcolorbox{aside}{asidestyle}
\newtcolorbox{exercises*}{exercisestyle}
\newtcolorbox{exercises}{exercisestyle2}
\newtcolorbox{exercisessec}{exercisestyle4}
\newtcolorbox{exercise}{exercisestyle3}
                      
\newenvironment{enumeratea}{
	\begin{enumerate}
	  \renewcommand{\labelenumi}{(\alph{enumi})}}
	{\end{enumerate}}

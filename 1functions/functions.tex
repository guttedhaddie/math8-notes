\graphicspath{{1functions/asy/}}
\thispagestyle{empty}

\title{Math 8 --- Functions and Modeling}
\author{Neil Donaldson}
\date{Spring 2024}
\maketitle	

\section*{Introduction}

The purpose of this course is provide some conceptual foundation to several mathematical topics commonly encountered at grade-school level. The job of a teacher is often one of \emph{selection}; choosing examples and explanations suited to the level and experience of your students. To do this effectively, you need to understand concepts at a higher level than you'll likely ever teach. Not all of our topics are central to the grade-school curriculum, and it is not our goal to teach you \emph{how} to teach, though the ideas and approaches we'll explore are in the most part suitable for a grade-school audience. As such, the \emph{mathematics} in this course shouldn't be difficult, requiring at most elementary calculus and a tiny bit of linear algebra; you should instead spend your time considering how to \emph{explain} the material, particularly to students with less mathematical knowledge than yourself.\par

\begin{minipage}[t]{0.72\linewidth}\vspace{0pt}
We start with two motivational problems.\footnotemark
\begin{enumerate}
  \item You wish to travel across the surface of a cube between two opposite vertices so that your path is as short as possible.\par
  Should you follow the path indicated?\par
  If yes, explain why.\par
  If not, how should you find the shortest path?
\end{enumerate}
\end{minipage}\hfill\begin{minipage}[t]{0.25\linewidth}\vspace{-10pt}
\flushright\includegraphics{intro-cube_0}
\end{minipage}\smallbreak

\begin{minipage}[t]{0.51\linewidth}\vspace{0pt}
\begin{enumerate}\setcounter{enumi}{1}
  \item Two houses are to be connected to the electricity supply using a single connection.\par
  How should we determine where to place the connection so as to minimize the required length of wire?\par
  What information do you need in order to find the connection point?
\end{enumerate}
\end{minipage}\hfill\begin{minipage}[t]{0.48\linewidth}\vspace{0pt}
\flushright\includegraphics[scale=0.9]{intro-elec}
\end{minipage}\medbreak

\footnotetext{We are grateful to materials from UT Austin's UTeach program for suggesting several of the examples in this course including these motivational problems.}

The point of these exercises isn't merely to find the right answer! Consider how you might discuss these problems with grade-school students of different levels. Why might calculus \emph{not} be a sensible approach? Are there commonalities between the two problems? Brainstorm some strategies and see if you can find different approaches.

%Unfold for the first then straight line. Reflection helps for the second: make equal angle, so split the distance between the houses in same ratio as the distances from the houses to the wire (similar triangles).

\goodbreak

\section{Sets \&\ Functions}

\subsection{Basic Definitions}

In this section we refresh the notion of a function. Consider how central this is to mathematics. Do you remember when you first encountered functions? How would you define \emph{function} to someone with limited mathematical knowledge? Would you use words like \emph{rule, assign, element, domain, vertical line test,} etc.? How helpful are these to your audience?

\begin{examples}{}{funcbasic}
How would you explain the idea that the following do or do not represent functions?\vspace{-5pt}
\begin{enumerate}\itemsep0pt
  \item $y=x^2$
  \item Mon: fish,\quad Tue: pork,\quad Wed: fajitas,\quad Thur: carbonara,\quad Fri: pizza
  \item $(3,5)$, $(2,6)$, $(4,2)$, $(3,1)$.
  \item $x^2=y^2$
\end{enumerate}
\end{examples}

After trying to explain these, perhaps you settle on a semi-formal definition such as the following:
\begin{quote}
A function $f$ is rule which assigns to each input $x$ exactly one output $f(x)$.
\end{quote}
Is this a useful definition? In what ways is it imprecise? How much does this matter?\medbreak

The answers to these questions plainly depend on your audience; from a formal palate a teacher should \emph{select} enough to convey what they think is important without overburdening and intimidating your students. As a teacher you need a more complete picture, which begins by thinking about what exactly are allowed to be \emph{inputs} and \emph{outputs.}

\begin{defn}{}{}
A \emph{set} $A$ is a collection of objects, known as \emph{elements.}\footnotemark{} The notation $a\in A$ means that $a$ is an element of $A$, sometime reads `$a$ lies in $A$.' Sets are usually written upper case and elements lower.\par
\begin{minipage}[t]{0.75\linewidth}\vspace{0pt}
A set $B$ is a \emph{subset} of a set $A$, written $B\subseteq A$, if every element of $B$ is also an element of $A$. That is:
\[b\in B\implies b\in A\]
In the picture, we have $B\subseteq A$, $a\in A$, $b\in B$ and $a\not\in B$.
\end{minipage}\hfill\begin{minipage}[t]{0.24\linewidth}\vspace{-12pt}
\flushright\includegraphics[scale=0.95]{functions-subset}
\end{minipage}
\end{defn}

\footnotetext{This is good enough for our purposes. It typically takes a term of set theory to convince you that this really isn't a good definition; selection again!}


\begin{examples}{}{}
\exstart Suppose the elements of a set $A$ are the numbers 1, 3, 5, 7 and 9. The simplest way to write this is in \emph{roster notation}: list the elements (in any order!) between curly brackets
\[A=\{1,3,5,7,9\}\]
\begin{enumerate}\setcounter{enumi}{1}
  \item[]$B=\{3,5,7\}$ is a subset of $A$. Subsets are commonly denoted using \emph{set-builder notation}:
	\[\{a\in A:\text{condition on $a$}\}\]
	For example, $B=\{a\in A:2<a<8\}$, ``The set $a$ in $A$ such that $a$ is strictly between 2 and 8." Can you express $B$ in other ways using set-builder notation? 
  \goodbreak
  
  \item You should be familiar with common sets of numbers: we use various combinations of roster and set-builder notation to informally define these.
  \begin{description}
		\item[Natural numbers] $\N=\{1,2,3,4,\ldots\}$. For instance $5\in\N$ but $-3\not\in\N$.
		\item[Integers] $\Z=\{\ldots,-2,-1,0,1,2,3,\ldots\}$. For instance $-4\in\Z$ but $\frac 45\not\in\Z$.
		\item[Rational numbers] or fractions: $\Q=\bigl\{\frac pq:p\in\Z,\ q\in\N\bigr\}$. For instance $-\frac 67\in\Q$; in this case $p=-6$ is an integer, and $q=7$ a natural number.
		\item[Real numbers] $\R$: for instance $\sqrt 2\in\R$. A formal definition is difficult, but you should be used to visualizing the real line. \emph{Intervals} are particularly important subsets, e.g.
		\[[-4,\pi)=\{x\in\R:-4\le x<\pi\}\]
		is a half-open interval.
		\item[Complex numbers] $\C=\bigl\{x+iy:x,y\in\R\bigr\}$.
	\end{description}
	You should be comfortable with the subset relationships between these sets
	\[\N\subseteq\Z\subseteq\Q\subseteq\R\subseteq\C\]
	You should have informally encountered the notion of \emph{irrationality}: for instance, $\sqrt 2$ and $\pi$ are real numbers but not rational.
\end{enumerate}
\end{examples}

\smallskip

The reason we need this language is that the inputs and outputs of a function are \emph{elements} of sets.

\begin{defn}{}{function}
The \emph{Cartesian product} of sets $A,B$ is the set of \emph{ordered pairs}
\[A\times B=\bigl\{(a,b):a\in A,b\in B\bigr\}\]
A \emph{function} from $A$ to $B$ is a non-empty subset $f\subseteq A\times B$ which satisfies the \emph{vertical line test}
\[\text{For each $a\in A$, there is a \emph{unique} $b\in B$ such that $(a,b)\in f$} \tag{$\ast$}\]
Rather than write $(a,b)\in f$, we use the more familiar notation
\[f:A\to B\quad\text{and}\quad f(a)=b\]
To a function $f:A\to B$ are associated three sets:\footnotemark
\begin{itemize}\itemsep0pt
  \item Domain: $\dom f=A$ is the set of \emph{inputs.}
  \item Codomain: $\operatorname{codom} f=B$ is the set of \emph{possible outputs.}
  \item Range: $\operatorname{range} f=\{b\in B:b=f(a)\text{ for some }a\in A\}$ is the set of \emph{realized outputs.}
\end{itemize}
We call $A$ the \emph{domain} and $B$ the \emph{codomain} of the function (the set of \emph{possible} outputs).
\end{defn}


\footnotetext{Even more formally, these sets make sense for any subset $f\subseteq A\times B$ (such is called a \emph{relation} from $A$ to $B$). For instance $\dom f=\{a\in A:(a,b)\in f\text{ for some }b\in B\}$. For a relation to be a function, we \emph{require} $\dom f=A$; this is the \emph{for each} part of $(\ast)$ in Definition \ref{defn:function}.}


Wow! Is the definition you should give to 10\th{} graders, or even to freshman calculus students?
\smallbreak

Think through each of Examples \ref{ex:funcbasic} in this formal context. To do this properly, we have to carefully label the constituent sets. For instance:
 
\begin{examples*}{\ref{ex:funcbasic} cont}{}
Revisit our earlier examples in this language. For number 2 we could choose
\begin{gather*}
A=\bigl\{\text{Mon, Tue, Wed, Thu, Fri}\bigr\},\qquad B=\bigl\{\text{carbonara, fajitas, fish, pizza, pork}\bigr\},\\[5pt]
f=\bigl\{(\text{Mon,\ fish}),\ (\text{Tue,\ pork}),\ (\text{Wed,\ fajitas}),\ (\text{Thu,\ carbonara}),\ (\text{Fri,\ pizza})\bigr\}
\end{gather*}
Is it clear where we made a \emph{choice}? What would be another choice? Try the other examples yourself.
\end{examples*}


\boldsubsubsection{Representing Functions}

You should be familiar with several ways in which we could represent a function.

\begin{example}{}{introfunction}
Consider the familiar \emph{formula} $f(x)=x^2$ in several contexts.
\begin{description}\itemsep8pt
  \begin{minipage}[t]{0.7\linewidth}\vspace{-5pt}
  	\item \emph{Table}\lstsp This presentation might be beneficial if the domain is a small set. For instance, the table on the right shows the situation with $\dom f=\{-1,0,1,2,3\}$ and $\operatorname{range}f=\{0,1,4,9\}$
  
  	\item \emph{Arrows}\lstsp A pictorial arrow diagram might also be helpful for illustrating situations with small domains.
  
  
  	\item \emph{Graph}\lstsp This is simply the set of ordered pairs $\bigl(x,f(x)\bigr)$: in the context of Definition \ref{defn:function}, \emph{the graph is the function!}\smallbreak
	 	Formulæ whose inputs and outputs are real numbers are common enough that two conventions are followed:
	 	\begin{itemize}
	 	  \item The \textcolor{red}{domain} is \emph{implied} to be all real numbers for which the formula makes sense: in this case all real numbers!
			\item The codomain is taken to be the set of real numbers.
		\end{itemize}
		In the absence of additional information, our formula implies that $f:\R\to\R$.\smallbreak
		
	The \textcolor{Brown}{range} of the function is the set of all possible outputs, in this case
	\[\operatorname{range}f=\{x^2:x\in\R\}=[0,\infty)\]
	is the half-open interval of non-negative real numbers.\smallbreak
		
	For `calculus-type' functions like these, the vertical line test really involves \textcolor{Green}{vertical lines}; every vertical line intersects the graph in exactly one point.\smallbreak
	In the picture, the \textcolor{orange}{dots} are the graph when the domain is the finite set $\{-1,0,1,2,3\}$.
  \end{minipage}\hfill\begin{minipage}[t]{0.29\linewidth}\vspace{0pt}
  	\flushright$\begin{array}{c|ccccc@{}}
	x&-1&0&1&2&3\\\hline
	f(x)&1&0&1&4&9
	\end{array}$\bigbreak
	\includegraphics[scale=0.95]{functions-quad2}\bigbreak
	\includegraphics[scale=0.95]{functions-quad}
  \end{minipage}
\end{description}
\end{example}

Other ways to represent functions are possible


\clearpage


\begin{exercises}{}{}
\exstart Let $d$ represent the cost in millions of dollars to produce $n$ cars, where $n$ is measured in 1000s. As clearly as you can, explain what is meant by $d(25)=431$. 

\begin{enumerate}\setcounter{enumi}{1}
	\item Temperature readings $T$ were recorded every two hours from midnight to noon. Time $t$ was measured in hours from midnight.
  \[\begin{array}{l|ccccccc}
  t&0&2&4&6&8&10&12\\\hline
  T\ (^\circ\mathrm{F})&82&75&74&75&84&90&93
  \end{array}\]
  \begin{enumerate}
    \item Use the readings to sketch a rough graph of $T$ as a function of $t$.
    \item Use your graph to estimate the temperature at 10:30\,a.m.
  \end{enumerate}
  
  
  \item State parts 1, 3 and 4 of Examples \ref{ex:funcbasic} using the formal language of Definition \ref{defn:function}. If you have a function, state the domain and range and explain how you know you have a function. If you don't have a function, explain why not.\par
  (\emph{There is no single correct answer to these, since insufficient information is provided})
  
  \item\begin{enumerate}
    \item Let $A=\{1,3,5,7,9\}$. Explain in words what is meant by the set $B=\{x\in A:x^2>10\}$, and state $B$ in roster notation.
    \item Find the set $C=\{x\in\N:(x-1)^2<16\}$ in roster notation.
    \item Find the Cartesian product $B\times C$ in roster notation. Is it the same as $C\times B$?
  \end{enumerate}
  
  \item Suppose that $f:\{-2,-1,0,1,2\}\to\R$ is defined by the formula $f(x)=x^3-4x+1$.\par
  Describe $f$ using a table, an arrow diagram and a graph.
  
  \item The unit circle is often represented by the implicit equation $x^2+y^2=1$.
  \begin{enumerate}
    \item Draw the circle and explain why the full circle isn't the graph of a function.
    \item Describe \emph{two} functions whose domains are the closed interval $[-1,1]$ and whose graphs together comprise the circle. What are the ranges of the two functions?
  \end{enumerate}
  
  
\end{enumerate}
\end{exercises}


\clearpage



\subsection{Linear Polynomials}

Perhaps the simplest functions are the \emph{linear polynomials}, so named because their graph is a straight line;\footnote{Don't confuse this with the meaning of a \emph{linear function} from linear algebra; in general $f(\lambda x)\neq \lambda f(x)$ for a linear polynomial.} $y=f(x)=mx+c$ where $m,c$ are constants. These functions make for very easy models: increase the input by $\Delta x$ and the output changes by $\Delta y=m\Delta x$ \emph{regardless} of the starting value $x$. 

\begin{example}[lower separated=false, sidebyside, sidebyside align=top seam, sidebyside gap=0pt, righthand width=0.35\linewidth]{}{}
Find the equation of the \textcolor{blue}{straight line} through the points $A=(1,3)$ and $B=(4,1)$.\smallbreak
Substitute both points into the equation and solve
\begin{align*}
\begin{cases}
3=m+c\\
1=4m+c
\end{cases}
&\implies -2=3m\\
&\implies m=-\frac 23,\quad c=3-m=\frac{11}3
\end{align*}
The \emph{\textcolor{Brown}{gradient}} or \textcolor{Brown}{\emph{slope} $m$} represents how far one climbs/falls on travelling one unit to the right.\smallbreak
The \textcolor{Green}{\emph{$y$-intercept} $c$} is the intersection of the line with the $y$-axis.
\tcblower
\flushright\includegraphics{line-line}
\end{example}

There is some bookkeeping to do here: how do we know that every straight line corresponds to such a linear function? This follows fairly easily from a useful fact regarding parametrizations.

\begin{thm}{}{lineparametrize}
The set of points on the line through distinct points $A$ and $B$ is
\[\ell_{A,B}=\bigl\{(1-t)A+tB:t\in\R\bigr\}\]
\end{thm}

\begin{proof}
There are several ways to think about this; we use what is essentially \emph{vector addition}.\par
\begin{minipage}[t]{0.6\linewidth}\vspace{-5pt}
Certainly the set of points
\[\ell_{O,B-A}=\{t(B-A):t\in\R\}\]
describes the straight line through the origin and the point $B-A$. We simply \textcolor{Brown}{shift} this line by $A$.\smallbreak
The locations of the points corresponding to various values of $t$ are \textcolor{Green}{marked}.
\end{minipage}\hfill\begin{minipage}[t]{0.39\linewidth}\vspace{0pt}
\flushright\includegraphics{line-line2}
\end{minipage}
\end{proof}

\begin{example}{}{}
The line through points $A=(3,6)$ and $B=(-1,4)$ may be parametrized by
\[(x,y)=(1-t)(3,6)+t(-1,4)=\bigl(3-4t,6-2t\bigr)\]
By solving for $t$ in terms of $x$, we see that this has equation
\[y=6-2t=6-2\cdot\frac{3-x}{4} =\frac 12x-\frac 92\]
\end{example}

\goodbreak

\begin{exercises}{}{}
	\exstart The cost of gasoline is \$4.20 per gallon on January 1\st{} and \$4.90 on March 1\st. State a \emph{linear} function for the cost of gasoline as a function of time.
%\par
%   (\emph{There are multiple correct answers, depending on your decision how to measure time}!)
\begin{enumerate}\setcounter{enumi}{1}
  \item You have a choice of three different cell-phone plans.
  \begin{enumerate}
    \item No monthly charge and 10\textcent{} per minute for all calls.
    \item \$10 per month and 5\textcent{} per minute for all calls.
    \item \$30 per month, regardless of how many calls you make.
  \end{enumerate}
  How should you determine which of the plans to purchase?
    
    
    \item Suppose $y=mx+c$ is the equation of a linear function. By choosing any two points $A,B$ on this line, find an explicit parametrization in the style of Theorem \ref{thm:lineparametrize}.
    
    \item Suppose $A=(x_0,y_0)$ and $B=(x_1,y_1)$ are given. If $x_1\neq x_0$, find the equation $y=mx+c$ of the line through these points.\par
    (\emph{You should recognize $m$ as the familiar `rise over run'})
    
    
    \item Suppose that a linear polynomial $f(x)=mx+c$ is also a \emph{linear function}:
    \[\text{For all }\lambda,x\in\R,\quad f(\lambda x)=\lambda f(x)\]
    What can you say about $f$?
  \end{enumerate}
\end{exercises}

% We can repeat such an analysis in general.
% \begin{itemize}
%   \item Given $y=mx+c$, let $A=(0,c)$ and $B=(1,m+c)$, the line through these consists of the points
% 	\[(x,y)=(1-t)A+tB=\bigl(t,(1-t)c+t(m+c)\bigr) =(t,mt+c)\]
% 	all of which satisfy $y=mx+c$.
% 	\item Conversely, the line through $A=(x_0,y_0)$ and $B=(x_1,y_1)$ is parametrized by
% 	\begin{gather*}
% 	(x,y)=(1-t)A+tB =\bigl(x_0+(x_1-x_0)t,y_0+(y_1-y_0)t \bigr)\\
% 	\implies y=y_0+(y_1-y_0)t=y_0+(y_1-y_0)\frac{x-x_0}{x_1-x_0} =\frac{y_1-y_0}{x_1-x_0}x +\frac{y_0x_1-x_0y_1}{x_1-x_0}
% 	\end{gather*}
% 	which has the form $y=mx+c$ with gradient $m=\frac{y_1-y_0}{x_1-x_0}$ given by the familiar `rise over run.'
% \end{itemize} 

% Homework get symmetric form. and see works for vertical lines
% In view of this discussion, a more symmetric formula for a straight line is available.
% 
% \begin{cor}{}{}
% The equation through $A=(x_0,x_1)$ and $B=(x_1,y_1)$ has equation
% \[((y_1\]
% \end{cor}

\clearpage

\subsection{Quadratic Polynomials}

Quadratic polynomials are functions of the form $y=f(x)=ax^2+bx+c$ where $a\neq 0$. The simplest is $y=x^2$, the standard parabola opening upwards. Here are commonly encountered activities:
\begin{enumerate}\itemsep0pt
  \item Find the \emph{roots/zeros} of $f$, the solutions to the equation $f(x)=0$.
  \item Sketch the \emph{graph} of the function $f$.
  \item Use quadratic functions to model a real-world problem.
\end{enumerate}

You likely know two methods for finding zeros: factorizing and the quadratic formula, each of which has its difficulties. With experience it is easy to spot that 
\[x^2+2x-15=(x-3)(x+5)=0\iff x=3\text{ or }x=-5\]
though this requires some creativity and can therefore be difficult for grade-school students, particularly when the coefficients are large. Students often prefer the quadratic formula since it always works, though at the cost of some intimidating algebra.
We'll think about factorization shortly. First, we see how the method of \emph{completing the square} lies behind both the quadratic formula and the standard approach to graphing quadratic functions.

\begin{example}{}{quadraticeasy}
Describe/graph the parabola $y=-3x^2+12x+4$.\par
\begin{minipage}[t]{0.62\linewidth}\vspace{-5pt}
Pay attention to the $x$ terms; $-3x^2+12x=-3(x^2-\textcolor{red}{4}x)$. Now
\[-3(x-\textcolor{red}{2})^2=-3(x^2-\textcolor{red}{4}x+4)=-3x^2+12x-12\]
gives most of what we want. Note how we \emph{divided the \textcolor{red}{$x$-coefficient} by two.} Now tidy up,
\[y=(-3x^2+12x-12)+16=-3(x-2)^2+16\]
\end{minipage}\hfill\begin{minipage}[t]{0.27\linewidth}\vspace{-15pt}
\flushright\includegraphics[scale=0.9]{poly-quad2}
\end{minipage}\medbreak
The parabola therefore opens downwards with its \textcolor{Green}{apex} at $(x,y)=(2,16)$.
\end{example}

This is easy to repeat in general:
\begin{align*}
ax^2+bx+c&=a\left(x^2+\frac bax\right)+c =a\left[\left(x+\frac b{2a}\right)^2-\frac{b^2}{4a^2}\right]+c\\
&=a\left(x+\frac b{2a}\right)^2-\frac{b^2-4ac}{4a}
\end{align*}
\begin{minipage}[t]{0.6\linewidth}\vspace{0pt}
The graph is that of the standard parabola which has been:
\begin{enumerate}\itemsep0pt
  \item Vertically scaled by $a$;
  \item Shifted horizontally so that the apex is at $x=-\frac b{2a}$;
  \item Shifted vertically by $\frac{4ac-b^2}{4a}$
\end{enumerate}
Completing the square therefore yields the \emph{quadratic formula.}
\end{minipage}\begin{minipage}[t]{0.4\linewidth}\vspace{-60pt}
\flushright\includegraphics[scale=0.9]{poly-quad}
\end{minipage}

\begin{thm}{}{}
If $a\neq 0$, then $ax^2+bx+c=0\iff x=\dfrac{-b\pm\sqrt{b^2-4ac}}{2a}$
\end{thm}

\goodbreak

\begin{example*}{\ref{ex:quadraticeasy} cont}{}
Our analysis suggests two methods for finding the \textcolor{purple}{roots}.
\begin{enumerate}
  \item Quadratic formula: with $a=-3$, $b=12$, $c=4$, we have
  \[x=\dfrac{-12\pm\sqrt{12^2-4(-3)\cdot 4}}{2(-3)} =\dfrac{-12\pm 4\sqrt{3^2+3}}{-6} =2\pm\dfrac{\sqrt{12}}{3} =2\pm\dfrac{2\sqrt{3}}{3}\]
  While it is always tempting to jump for a formula, it often leads to difficult surd expressions. We simplified by noticing the common factor of $4^2$ inside the square root. Without this, we'd be faced with $\sqrt{144+48}=\sqrt{192}$.
  \item Use the fact that we've already completed the square:
  \[-3(x-2)^2+16=0\iff (x-2)^2=\frac{16}3\iff x=2\pm\frac 4{\sqrt 3}\]
  In many cases it is simpler to complete the square than to use the quadratic formula---remember that they are equivalent!
\end{enumerate}
\end{example*}


Polynomials are often employed in modelling due to their simplicity and ease of evaluation. As you saw in calculus, the motion of a falling body, or of any projectile can be modelled using quadratic polynomials, an observation going back to at least to Galileo in the early 1600s: the distance travelled by a falling body is proportional to the \emph{square} of the time taken $y(t)-y(0)\propto t^2$.

\begin{example}{}{quadseqgraph}
A body is dropped from a height of 125 meters, taking exactly 5 seconds to reach the ground. Its height at time $t$ seconds is given by $y(t)=125-5t^2$\,m.\smallbreak
\begin{minipage}[t]{0.68\linewidth}\vspace{-8pt}
This certainly fits Galileo's observation: $y(t)-y(0)=-5t^2$ is indeed proportional to $t^2$.\smallbreak
Over each interval of 1\,s, we may ask how far the body falls; we summarize in a table.
\[
\begin{array}{c|ccccccccccc}
t&0&&1&&2&&3&&4&&5\\\hline
y(t)&125&&120&&105&&80&&45&&0\\\hline
y(t)-y(0)&0&&-5&&-20&&-45&&-80&&-125\\\hline
\Delta y&&\makebox[0pt][c]{$-5$}&&\makebox[0pt][c]{$-15$}&&\makebox[0pt][c]{$-25$}&&\makebox[0pt][c]{$-35$}&&\makebox[0pt][c]{$-45$}
\end{array}
\]
\end{minipage}\hfill\begin{minipage}[t]{0.29\linewidth}\vspace{-25pt}
\flushright\includegraphics{poly-quad5}
\end{minipage}\medbreak
Since each interval has duration 1\,s, each $\Delta y$ is the \emph{average speed} of the falling body over that interval.
\end{example}

You'll have seen problems like this in calculus; likely you want to \emph{differentiate} to find the \emph{velocity} $y(t)=-10t$\,m/s and \emph{acceleration} $y''(t)=-10$\,m/s$^2$. However, historically and in introductory calculus, it is problems like these that \emph{motivate the definition} of the derivative: the last line of the table really does suggest that speed is a linear function!\par
Armed with calculus, Galileo's observation is indeed that the height $y(t)$ solves the differential equation
\[\diff[^2y]{t^2}=-g\]
where $g$ is the constant acceleration due to gravity; approximately 32\,ft/s$^2$ or 10\,m/s$^2$. Though unless you are explicitly teaching calculus, or Newtonian Physics, this is probably a bad place to start! 



\begin{example}[lower separated=false, sidebyside, sidebyside align=top seam, sidebyside gap=0pt, righthand width=0.3\linewidth]{}{frisbee}
Your frisbee is stuck 15\,m up a tree. Standing 10\,m away from the base, you throw a ball with the intent of knocking the frisbee out of the tree.\smallbreak
The standard approach to modelling such problems involves considering the horizontal and vertical motions separately.
\begin{quote}
\emph{Horizontal}\lstsp $x(t)=pt+q$ is a \emph{linear function} of time.\smallbreak
\emph{Vertical}\lstsp $y(t)=-10t^2+rt+s$ is a \emph{quadratic function} of time.
\end{quote}
Substituting for $t$ yields a quadratic function for the \textcolor{blue}{trajectory}
\[y(x)=ax^2+bx+c\]
We'll leave the details of the solution of this problem to Exercise \ref{exs:frisbee}. For the present, consider why there are \emph{multiple answers}; can you explain why \emph{without} explicitly solving the problem?
\tcblower
\flushright\includegraphics[scale=0.88]{tree1}
\end{example}


\begin{exercises}{}{}
\exstart Complete the square for each quadratic function, use it to find the range and to graph the function.
\begin{enumerate}\setcounter{enumi}{1}  
  \item[]\begin{enumerate}
    \item $f(x)=x^2-6x+5$ %$(x-3)^2-4$
    \item $f(x)=-x^2+x+1$ %$-(x-\frac 12)^2+\frac 54$
    \item $f(x)=-3x^2+8x+5$ %$3(x+\frac 43)^2-\frac 13$
  \end{enumerate}
  %For part (a), also find two intervals $(-\infty,k]$ and $[k,\infty)$ (same $k$!) on which $f$ is invertible. For each interval, compute the inverse function $f^{-1}$.
  
  
  \item For the quadratic function $y=2x^2-5x+7$, produce a table for $x\in\{0,1,2,3,4,5,6\}$ similarly to that in Example \ref{ex:quadseqgraph}. What do you observe about $\Delta y$?
  
  
	\item\begin{enumerate}
	  \item Find the equations of all quadratic polynomial functions which pass through the points $(1,3)$ and $(2,4)$.
		\item More generally, if $P=(a,b)$ and $Q=(c,d)$ are given, where $c\neq a$, find all quadratic functions whose graphs contain $P$ and $Q$.
	\end{enumerate}
  
  \item\label{exs:frisbee} Consider the frisbee/tree problem (Example \ref{ex:frisbee}).
  \begin{enumerate}
    \item Assume that you're standing at the origin and the frisbee is at the point $(10,15)$. Find all trajectories.
    \item (Hard)\lstsp Find a formula linking the initial speed and gradient of the parabola (the initial speed and direction in which you throw the ball).
    \begin{enumerate}
			\item If you throw the ball in such a way that the initial \emph{vertical} speed of the ball is twice its \emph{horizontal} speed, find how fast you have to throw the ball in order to hit the frisbee.
      \item What is the \emph{minimum} speed at which you could throw the ball if you want to dislodge the frisbee?
    \end{enumerate}
    (\emph{Hint: You'll need some calculus! In the language of the original problem, the initial slope is $m=\frac rp$ and speed $v=\sqrt{p^2+r^2}$; why?})
	\end{enumerate}


\end{enumerate}
\end{exercises}

\clearpage



\subsection{Polynomials, Factorization \& the Rational Roots Theorem}

Recall our simple example of factorization in the previous section
\[x^2+2x-15=(x-3)(x+5)=0\iff x=3\text{ or }x=-5\]
That this approach provides \emph{all} roots depends on several familiar algebraic facts:
\begin{enumerate}
  \item Factor Theorem: $f(c)=0\iff x-c$ is a \emph{factor} of $f(x)$.
  \item No zero-divisors: $pq=0\iff p=0$ or $q=0$. 
  \item A quadratic has \emph{at most two} distinct roots.
\end{enumerate}
We'll examine this more closely at the end of this section. However, for students first learning about factorization, it isn't the \emph{why} that's the challenge, it's the \emph{how.} Multiplying out $(x-3)(x+5)$ is mechanical, but factorizing requires some creativity; we can't really factor without somehow knowing that 3 and $-5$ are roots! Beyond making a lucky guess, how do we go about this? 

\begin{example}{}{ratroots1}
Let's re-examine $f(x)=x^2+2x-15=0$ in a couple of stages.
\begin{description}
\item[\normalfont\emph{Integer solutions}] The simplest type of root would be an \emph{integer} $n$. If $f(n)=0$, observe that
\begin{align*}
n^2+2n-15=0&\implies n(n+2)=15\implies \text{15 is divisible by $n$}\\
&\implies n=\pm 1,\pm 3,\pm 5,\pm 15
\end{align*}
There are only \emph{eight possible candidates.} It doesn't take long to test all of them:
\[
\begin{array}{c|cccccccc}
n&1&-1&\textcolor{red}{3}&-3&5&\textcolor{red}{-5}&15&-15\\\hline
f(n)&-12&-17&\textcolor{red}{0}&-12&20&\textcolor{red}{0}&240&180
\end{array}
\]
The two integer solutions are therefore $x=3$ and $x=-5$.
\item[\normalfont\emph{Rational Solutions}] If you believe that a quadratic polynomial has \emph{at most} two solutions, then you're done. The next simplest possibility, however, is that a solution be a \emph{rational number} $x=\frac pq$ where we may assume this is in \emph{lowest terms}.\footnotemark{} Substituting into the polynomial, we see that
\[\frac{p^2}{q^2}+2\frac pq-15=0\iff p^2+2pq-15q^2=0\]
Remembering that $p,q$ are \emph{integers,} we rearrange this equation in two ways:
\begin{description}
  \item[$\textcolor{blue}{p}(p+2q)=15q^2$]\lstsp Since the \textcolor{blue}{left side} is a multiple of $p$, so also is the \emph{right.} Since $p,q$ have no common factors, it follows that $p$ divides into 15 (15 is a multiple of $p$).
  \item[$p^2=\textcolor{red}{q}(15q-2p)$]\lstsp Since the \textcolor{red}{right side} is a multiple of $q$, so also is the \emph{left.} Since $p,q$ have no common factors, we conclude that $q=1$.
\end{description}
The upshot is that the only rational solutions to $f(x)=0$ are the two \emph{integers} we've already found!
\end{description}
\end{example}

\footnotetext{I.e.{} $p\in\Z$ and $q\in\N$ have no common factors: $\gcd(p,q)=1$.}

\goodbreak

\begin{defn}{}{}
A \emph{degree $n$ polynomial} is any function of the form
\[f(x)=a_nx^n+a_{n-1}x^{n-1}+\cdots+a_1x+a_0\]
where the \emph{coefficients} $a_k$ are constants with $a_n\neq 0$.
\end{defn}

A quadratic polynomial has degree 2 and a linear polynomial $mx+c$ degree one\footnote{A non-zero constant polynomial has degree zero. Convention is for the \emph{zero polynomial} $y\equiv 0$ to have degree $-\infty$, so that the theorem $\operatorname{deg} fg=\operatorname{deg} f+\operatorname{deg} g$ holds for all polynomials.} (if $m\neq 0$).\smallbreak

Our analysis in Example \ref{ex:ratroots1} is easily generalized in a famous result.

\begin{thm}{Rational Roots}{}
Suppose $f(x)=\textcolor{blue}{a_n}x^n+\cdots+\textcolor{red}{a_0}$ has \emph{integer} coefficients where $a_n$ and $a_0$ are non-zero. If $x=\frac pq$ is a rational root in lowest terms, then $\textcolor{blue}{q}$ divides into $\textcolor{blue}{a_n}$ and $\textcolor{red}{p}$ divides into $\textcolor{red}{a_0}$.
\end{thm}

\begin{proof}
Substitute into the function and multiply by $q^n$ to obtain an equation where everything is an \emph{integer}\vspace{-10pt}
\[\underbrace{a_np^n+a_{n-1}p^{n-1}q+\cdots+a_1pq^{n-1}}_{\text{divisible by $\textcolor{red}{p}$}} \hspace{-129pt} \overbrace{\phantom{a_{n-1}p^{n-1}q+\cdots+a_1pq^{n-1}}+a_0q^n}^{\text{divisible by $\textcolor{blue}{q}$}}=0\]
By considering the braced terms and recalling that $p,q$ have no common factors, we conclude that $a_n$ is divisible by $q$ and $a_0$ by $p$.
\end{proof}


\begin{examples}{}{}
\exstart If $x=\frac pq$ is a rational root in lowest terms of $f(x)=2x^2-x-3$, then $q\in\{1,2\}$ and $p\in{\pm 1,\pm 3}$. The possibilities are therefore
\[x\in\bigl\{\pm 1,\pm 3,\pm\tfrac 12,\pm\tfrac 32\bigr\}\]
all of which are easily checked:
\[
\begin{array}{c|cccccccc}
x&1&\textcolor{red}{-1}&3&-3&\frac 12&-\frac 12&\textcolor{red}{\frac 32}&-\frac 32\\\hline
f(x)&-2&\textcolor{red}{0}&12&18&-3&-2&\textcolor{red}{0}&3
\end{array}
\]
The two roots are indicated and the polynomial can be factorized $f(x)=(2x-3)(x+1)$.
\begin{enumerate}\setcounter{enumi}{1}
  \item If the cubic polynomial $f(x)=x^3-2x^2+5$ had any rational roots, the only  possibilities would be $\pm 1, \pm 5$. However none of these work,
  \[f(1)=4,\quad f(-1)=2,\quad f(5)=80,\quad f(-5)=-170\]
  whence $f(x)=0$ has no rational roots.
\end{enumerate}
\end{examples}

Unless there are very few candidates, it can be time-consuming to check them all by hand. Moreover, unless you find $n$ distinct rational solutions, you still don't know that you've found everything. The rational roots theorem is therefore typically used together with factorization; it really just gives you some options for where to start. This still isn't easy, as the next example shows.
\goodbreak

\begin{example}{}{factoreasy}
Consider the cubic function $f(x)=x^3-x^2-7x+10$. The rational roots theorem gives us eight candidates: $x=\pm 1,\pm 2,\pm 5,\pm 10$. It is not difficult to try the first few of these in your head: by inspection, we see that
\[f(2)=8-4-14+10=0\]
and the factor theorem says that $x-2$ must be a factor. The factorization can be performed in various ways. Here are three options, though all are essentially versions of the same process.
\begin{description}
	\item[\normalfont\emph{Long division}]\lstsp You should have practiced this in high-school!
  \[\polylongdiv{x^3-x^2-7x+10}{x-2} \implies x^3-x^2-7x+10=(x-2)(x^2+x-5)\]
  \item[\normalfont\emph{Multiply out and solve}]\lstsp We know that $f(x)=(x-2)q(x)$ where $q(x)$ is some quadratic polynomial. Thus let $q(x)=ax^2+bx+c$ and multiply out:
  \[x^3-x^2-7x+10=(x-2)(ax^2+bx+c) =ax^3+(b-2a)x^2+(c-2b)x-2c\]
  Equating coefficients, we obtain the same factorization as before,
  \[a=1,\quad b=-1+2a=1,\quad c=\frac{10}{-2}=-5\]
  \item[\normalfont\emph{Term-by-term factorization}]\lstsp With practice you can factorize in one line with no working!
  \begin{enumeratea}
    \item The first term must be $\textcolor{red}{x^2}$ to create $\textcolor{red}{x^3}$
    \[\textcolor{red}{x^3}\textcolor{blue}{\,-\,x^2}\textcolor{Green}{\,-\,7x}\textcolor{purple}{\,+\,10}=(x-2)(\textcolor{red}{x^2}\,+\cdots)=\textcolor{red}{x^3}\,-2x^2+\cdots\]
    \item To correct the \textcolor{blue}{$x^2$} term, add $\textcolor{Green}{x}$:
    \[(x-2)(x^2+\,\textcolor{Green}{x}\,+\cdots)=\textcolor{red}{x^3}\textcolor{blue}{\,-\,x^2}\,\textcolor{Green}{-\,2x}+\cdots\]
    \item To correct the \textcolor{Green}{$x$} term, subtract $\textcolor{purple}{5}$:
    \[(x-2)(x^2+x-\textcolor{purple}{\,5})=\textcolor{red}{x^3}\textcolor{blue}{\,-\,x^2}\,\textcolor{Green}{-\,7x}\textcolor{purple}{\,+\,10}\]
    \item Since the last term $\textcolor{purple}{10}$ is correct, the factorization worked!
	\end{enumeratea}
\end{description}
	
	You might have seen other approaches involving arranging the coefficients in a table; the calculations required to complete the table are \emph{exactly} those seen above; all these methods are the same!
\end{example}

\goodbreak

\boldsubsubsection{Why Does Factorization Work?}


The theory behind factorization relies on some algebra. Here is a \emph{brief} treatment.

\begin{thm}{Factor Theorem}{}
Suppose $f(x)$ is a degree $n$ polynomial. Then:
\begin{enumerate}
  \item A value $c$ is a root if and only if $f(x)=(x-c)q(x)$ for some (degree $n-1$) polynomial $q(x)$.
  \item The polynomial has \emph{at most} $n$ distinct roots.
\end{enumerate} 
\end{thm}

\begin{proof}
\begin{enumerate}
  \item ($\Leftarrow$)\lstsp This is essentially trivial: $f(x)=(x-c)q(x)\implies f(c)=(c-c)q(c)=0$.\smallbreak
($\Rightarrow$)\lstsp This relies on the \emph{division algorithm for polynomials}: if $f,g$ are polynomials, then there exist unique polynomials $q,r$ with\footnotemark{}
% \footnote{Let $S=\{f(x)-g(x)q(x):q(x)\in\R[x]\}$. If $0\in S$  stop; $g$ divides $f$. Otherwise, let $k=\min S$ (exists by well-ordering) and let $r(x)=f(x)-g(x)q(x)=a_kx^k+\cdots$ have degree $k$. If $g(x)=b_jx^j+\cdots$ has degree $j\le k$, observe that
% \[f(x)-g(x)\bigl(q(x)+\tfrac{a_k}{b_j}x^{k-j}\bigr)=r(x)-a_kx^k-\cdots\]
% has degree $<k=\deg r$: contradiction. Thus $\deg r<\deg g$.\smallbreak
% For uniqueness, suppose had two candidates $r_1,r_2$, then
% \[r_1-r_2=g(q_2-q_1)\]
% But $\deg(r_1-r_2)<\deg g$ and $\deg(g(q_2-q_1))\ge \deg g$ is a contradiction unless $q_1=q_2$ and both sides are zero.}
\[f(x)=g(x)q(x)+r(x)\quad \text{and}\quad \deg r<\deg g\]
In the special case where $g(x)=x-c$ is linear, then $r(x)$ must be a constant and so
\[f(x)=(x-c)q(x)+f(c)\]
\item Suppose $c_1,\ldots,c_n$ are distinct real roots. By part 1, $f(x)=(x-c_1)q_1(x)$. Since
\[0=f(c_2)=(c_2-c_1)q_1(c_2)\implies q_1(c_2)=0\]
we may then factor $x-c_2$ out of $q_1(x)$ to obtain
\[f(x)=(x-c_1)(x-c_2)q_2(x)\]
Repeat this process to factor out all $n$ linear polynomials $x-c_k$:
\[f(x)=(x-c_1)\cdots(x-c_n)q_n\]
where $q_n\neq 0$ is \emph{constant.} Plainly $f(c)=(c-c_1)\cdots(c-c_n)q_n=0\iff c=c_j$  for some $j$, so there are no other roots.\qedhere
\end{enumerate}
\end{proof}

\footnotetext{For a given example, $q,r$ may be found by long-division. This is similar (and may be proved similarly) to the more familiar division algorithm for integers: if $m,n$ are integers, then there exist unique integers $q,r$ for which
\[m=qn+r\ \text{ and }\ 0\le r<\nm n\]
In grade-school, this is typically written $m\div n=q\,\textsf{r}\,r$ \ ($q$ remainder $r$); e.g.{} $23\div 4=5\,\textsf{r}\,3$ corresponds to $23=5\times 4+3$.}

\begin{example*}{\ref{ex:factoreasy} cont}{}
We know that $f(x)=x^3-x^2-7x+10=(x-2)(x^2+x-5)$. But then
\[f(x)=0\iff x-2=0\ \text{ or }\ x^2+x-5=0\]
The former gives the root $x=2$, and the latter can be attacked via the quadratic formula or completing the square; the polynomial therefore has exactly three roots
\[x=2,\frac{-1\pm\sqrt{21}}2\] 
\end{example*}


\begin{example}{}{}
We finish with a quick example of how long division (or any other factorization method as in Example \ref{ex:factoreasy}) computes the ingredients in the division algorithm.\smallbreak
If $f(x)=x^3+7x^2-2$ and $g(x)=x^2-2$, then\par
\begin{minipage}[t]{0.35\linewidth}\vspace{-13pt}
\[\polylongdiv{x^3+7x^2-2}{x^2-2}\]
\end{minipage}\hfill\begin{minipage}[t]{0.6\linewidth}\vspace{10pt}
$\negthickspace\implies x^3+7x^2-2=(x^2-2)(x+7)+(2x+12)$\medbreak
Otherwise said, $f(x)=g(x)q(x)+r(x)$, where\medbreak
$q(x)=x+7$, \ $r(x)=2x+12$ \ and \ $\deg r=1<2=\deg g$.
\end{minipage}
\end{example}

\goodbreak



\begin{exercises}{}{}
\exstart Apply the rational root theorem to the polynomial $x^3+2x^2-x-2$ and use it to factorize the polynomial.
\begin{enumerate}\setcounter{enumi}{1}
  \item Repeat the previous question for the polynomial $6x^2+x-2$.

	\item Use the rational roots theorem to prove that the polynomial $2x^5-3x+7$ has no rational roots.
  
  \item Factorize the following polynomials and thereby find their (real) roots. Explain your steps carefully.
  \begin{enumerate}
    \item $f(x)=x^3+2x^2-3x$ %$(x-1)(x+3)x$
    \item $f(x)=x^4-13x^2+36$ %$(x^2-9)(x^2-4)$
    \item $f(x)=x^3-7x-6$ %(x+2)(x^2-2x-3)
  \end{enumerate}
  
  
  \item Show that the polynomial $f(x)=x^6-2x^5-x^4-4x^3-4x^2-4x-6$ %$=(x+1)(x^5-3x^4+2x^3-6x^2+2x-6)=(x-3)(x+1)(x^4+2x^2+2)$
  has exactly two real roots by factorizing it.
	
	\item The polynomial $f(x)=2x^4-3x^3+2x^2+3x-9$ has only one rational root. Find it and factorize the polynomial as $f(x)=g(x)q(x)$ where $\deg g=1$.
	
	
	\item Find unique polynomials $q(x)$ and $r(x)$ for which $f(x)=g(x)q(x)+r(x)$ and $\deg r<\deg g$.
	\begin{enumerate}
	  \item $f(x)=x^3+1$ and $g(x)=x+2$.
	  \item	$f(x)=x^4+x^3-2$ and $g(x)=x^2+1$.
	\end{enumerate}  

  \item Let $f(x)=ax^3+bx^2+cx+d$ be a cubic polynomial. `Complete the cube' by finding a constant $k$ such that
  \[f(x)=a(x-k)^3+p(x-k)+q\]
  has no $(x-k)^2$ term (here $p,q$ are constants).\par
  (\emph{Hint: evaluate $f(x+k)$})
  
  \item Suppose that $\deg f=k$ and $\deg g=l$.
  \begin{enumerate}
    \item Show that $\deg(fg)=kl$.
    \item Is it always the case that $\deg(f+g)=\max(k,l)$? Why/why not?
  \end{enumerate}
\end{enumerate}
\end{exercises}

\clearpage


\subsection{Inverse Functions \& the Horizontal Line Test}

The informal idea of an inverse function is that $f^{-1}$ takes the \emph{output} of $f$ and returns its \emph{input} (and vice versa).

\begin{example}{}{}
Define a simple function using a table or an arrow diagram\par
\begin{minipage}[t]{0.74\linewidth}\vspace{-17pt}
\[\def\arraystretch{1.1}\begin{array}{c|cccc}
x&1&2&3&4\\\hline
f(x)&4&2&5&7
\end{array}\qquad\begin{array}{c|cccc}
y&4&2&5&7\\\hline
f^{-1}(y)&1&2&3&4
\end{array}
\]
Its inverse $f^{-1}$ is plainly the function obtained by \emph{reversing the arrows} or flipping the table upside-down.
\end{minipage}\hfill\begin{minipage}[t]{0.25\linewidth}\vspace{-17pt}
\flushright\includegraphics{inverse-easy}
\end{minipage}
% \par
% \[f^{-1}\bigl(f(x)\bigr)=x\quad\text{and}\quad f\bigl(f^{-1}(y)\bigr)=y\tag{$\ast$}\]
% for all possible inputs $x,y$ to either function, so inputs are indeed recovered from outputs.
\end{example}

\begin{defn}{}{inversefn}
A function $f:A\to B$ is \emph{invertible} if it has an \emph{inverse function}: a function $f^{-1}:B\to A$ for which
\[f^{-1}\bigl(f(x)\bigr)=x\quad\text{and}\quad f\bigl(f^{-1}(y)\bigr)=y\tag{$\ast$}\]
for all possible inputs $x\in A$ and $y\in B$.
\end{defn}

Certainly the above example satisfies the input--output properties ($\ast$). Our main concern is identifying when a function is invertible, how to make it so if not, and how to find an inverse. To motivate this, we consider two simple examples.



\begin{examples}{}{invexs}
%\exstart Define a simple function using a table or an arrow diagram
% \begin{enumerate}\setcounter{enumi}{1}
% \begin{minipage}[t]{0.74\linewidth}\vspace{-17pt}
% \item[]\[\def\arraystretch{1.1}\begin{array}{c|cccc}
% x&1&2&3&4\\\hline
% f(x)&4&2&5&7
% \end{array}\qquad\begin{array}{c|cccc}
% y&4&2&5&7\\\hline
% f^{-1}(y)&1&2&3&4
% \end{array}
% \]
% Its inverse $f^{-1}$ is plainly the function obtained by \emph{reversing the arrows} or flipping the table upside-down. Certainly\vspace{-5pt}
% \end{minipage}\hfill\begin{minipage}[t]{0.25\linewidth}\vspace{-17pt}
% \flushright\includegraphics{inverse-easy}
% \end{minipage}\par
% \[f^{-1}\bigl(f(x)\bigr)=x\quad\text{and}\quad f\bigl(f^{-1}(y)\bigr)=y\tag{$\ast$}\]
% for all possible inputs $x,y$ to either function, so inputs are indeed recovered from outputs.
\exstart The function $f(x)=2x$ has inverse $f^{-1}(y)=\frac y2$.
\begin{enumerate}\setcounter{enumi}{1}
\begin{minipage}[t]{0.79\linewidth}\vspace{-10pt}
	\item[]Again our input--output conditions $(\ast)$ are satisfied.\smallbreak
	%These functions are so straightforward that it isn't worth mentioning domains and ranges: in both cases they may be assumed to be the entire real line $\R$.\smallbreak
	The \textcolor{blue}{graph} admits an interpretation of $f^{-1}$ similar to the arrow diagram.
	\begin{itemize}\itemsep2pt
	  \item The function $f$ takes an input $x$, moves it \textcolor{Green}{vertically} to the graph and then \textcolor{Green}{projects} to the $y$-axis. This interpretation is precisely the vertical line test (Definition \ref{defn:function})!
	  \item The inverse function \emph{reverses the arrows}: transport an input $y$ \textcolor{orange}{horizontally} to the graph and then \textcolor{orange}{project} to the $x$-axis.
	\end{itemize}
\end{minipage}\hfill\begin{minipage}[t]{0.2\linewidth}\vspace{-25pt}
\flushright\includegraphics{inverses-line}
\end{minipage}\smallbreak

	
\begin{minipage}[t]{0.71\linewidth}\vspace{-5pt}
	\item\label{ex:invexs1} The graph of the function $f(x)=x^2-1$ is drawn. This time, when attempting to move a real number $y$ horizontally to the graph, we usually encounter one of two problems:
	\begin{enumerate}\itemsep2pt
	  \item If $y>-1$, we have \textcolor{orange}{two choices}.
	  \item If $y<-1$, there is \textcolor{Purple}{no intersection}.
	\end{enumerate}
	The naïve approach of \emph{reversing the arrows} is insufficient to define an inverse. However, a simple remedy arises by staring at the graph:
	\begin{itemize}\itemsep2pt
	  \item Problem (a) goes away if we delete the \textcolor{blue}{left half} of the graph. Equivalently, we \emph{restrict the domain} of $f$ to $[0,\infty)$. 
	  \item Problem (b) disappears if we insist that $y\ge-1$. Equivalently, we \emph{restrict the codomain} of $f$ to its \emph{range} $[-1,\infty)$. 
	\end{itemize}
\end{minipage}\hfill\begin{minipage}[t]{0.28\linewidth}\vspace{0pt}
\flushright\includegraphics{inverses-quad}
\end{minipage}\medbreak	
	After making these restrictions so that $f:[0,\infty)\to [-1,\infty)$, it is now easily checked that
	\[f^{-1}(y)=\sqrt{y+1},\quad f^{-1}:[-1,\infty)\to[0,\infty)\]
	satisfies our input--output conditions $(\ast)$ and is therefore the inverse of $f$:
	\begin{gather*}
	x\in [0,\infty)\implies f^{-1}\bigl(f(x)\bigr) =\sqrt{(x^2-1)+1}=x\\
	y\in[-1,\infty)\implies f\bigl(f^{-1}(y)\bigr)=\bigl(\sqrt{y+1}\bigr)^2-1=y
	\end{gather*}
\end{enumerate}
\end{examples}


\boldinline{What makes a function invertible?}

The fixes in our last example can be rephrased succinctly:
% \begin{quote}
\begin{center}
$\tcbhighmath{\text{
\emph{Horizontal line test}: every horizontal line must intersect the graph \emph{exactly once}}}$
\end{center}
% \end{quote}
This unpacks to two conditions, each of which addresses one of the problems seen in the example.

\begin{defn}{}{}
Let $f:A\to B$ be a function. We say that $f$ is:
\begin{enumeratea}
\item \emph{1--1/one-to-one} if distinct inputs $x_1\neq x_2\in A$ have distinct outputs $f(x_1)\neq f(x_2)$. Equivalently,
  \[\text{If $x_1,x_2\in A$, then }\ f(x_1)=f(x_2)\implies x_1=x_2\]
  If $A,B$ are sets of real numbers, each horizontal line intersects the graph \textcolor{orange}{at most once}.
  \item \emph{Onto} if $\operatorname{range}(f)=B$. Equivalently,
  \[\text{Given $y\in B$, there is some $x\in A$ for which $y=f(x)$}\]
  If $A,B\subseteq\R$, the horizontal line through $y\in B$ intersects the graph \textcolor{Purple}{at least once}.
\end{enumeratea}
\end{defn}

Putting these ideas together, we see that a function is 1--1 and onto if and only if every $y\in B$ corresponds to a \emph{unique} $x\in A$ for which $y=f(x)$. In summary:

\begin{thm}{}{}
$f:A\to B$ is invertible if and only if it is 1--1 and onto. Its inverse is the function $f^{-1}:B\to A$ such that $f^{-1}(y)=x$ whenever $y=f(x)$.
\end{thm}

%Let us revisit our previous example in this context.

\begin{example*}{\ref*{ex:invexs}.\ref{ex:invexs1}, mk.\,II}{}
Consider the properties:
\begin{enumeratea}
	\item $f(x_1)=f(x_2)\implies x_1^2-1=x_2^2-1\implies x_1^2=x_2^2\implies x_1=\pm x_2$.\par
	To force $f$ to be 1--1, it is enough to \emph{restrict the domain} so that all $x$ have the same sign: the obvious choice is $\dom(f)=[0,\infty)$.
	\item $\operatorname{range}(f)=\{x^2-1:x\in[0,\infty)\}=[-1,\infty)$. We force $f$ to be onto by \emph{restricting its codomain} to $[-1,\infty)$.
\end{enumeratea}
The inverse function is obtained by solving $y=x^2-1$ for $x$:
\[x^2=y+1\implies x=f^{-1}(y)=\sqrt{y+1}\]
where we use the \emph{positive square root} since $x\in[0,\infty)=\dom(f)$.
\end{example*}
\goodbreak


\boldinline{An algorithm for inverting functions}

Our discussion provides an algorithmic process for making a function $f:A\to B$ invertible and finding an inverse.
\begin{enumeratea}\itemsep2pt
	\item Check that $f$ is 1--1. If not, \emph{restrict the domain} until it is.
	\item Check that $f$ is onto. If not, \emph{redefine} $B=\operatorname{range}(f)$.
	\item Solve $y=f(x)$ for $x=f^{-1}(y)$.
\end{enumeratea}

Since we typically prefer $x$ as an input, it is common also to \emph{switch} $x,y$ at the end so that $y=f^{-1}(x)$. If $A,B\subseteq\R$, this switch is equivalent to \emph{reflecting the graph} in the line $y=x$.\smallbreak

Note also that step (a) might involve some \emph{choice}; depending on how you restrict the domain, you can obtain multiple inverse functions! To see this in action, we return once more to our favorite example.

\begin{example*}{\ref*{ex:invexs}.\ref{ex:invexs1}, mk.\,III}{}
Recall that if $f(x)=x^2-1$, then
\[f(x_1)=f(x_2)\implies x_1=\pm x_2\]
Instead of restricting the domain to $[0,\infty)$, we can force $f$ to be 1--1 by \emph{choosing} $\dom(f)=(-\infty,0]$. The range/codomain remains $[-1,\infty)$, but the inverse function is now different:
\[x^2=y+1\implies x=-\sqrt{y+1}\in (-\infty,0]=\dom(f) \implies f^{-1}(x)=-\sqrt{x+1}\]
This time the new domain for $f$ forced us to use the \emph{negative square root.}
\begin{center}
\begin{minipage}[t]{0.52\linewidth}\vspace{30pt}
	\centering\includegraphics{inverses-combi}
\end{minipage}\begin{minipage}[t]{0.35\linewidth}\vspace{0pt}
	\centering\begin{tabular}{@{}c@{}}
		\includegraphics{inverses-combi2}\\[25pt]
		\includegraphics{inverses-combi3}
	\end{tabular}
\end{minipage}
\end{center}
There is nothing stopping us from choosing other domains on which $f$ is 1--1, but these are certainly the most natural choices.
\end{example*}

\clearpage


\iffalse
\clearpage

\boldsubsubsection{OLD Version}

You've long been used to the idea of inverse functions: $g=f^{-1}$ means that $g$ takes the \emph{output} of $f$ and returns the \emph{input,} and vice versa.

\begin{defn}{}{inversefn}
Functions $f:A\to B$ and $g:B\to A$ are \emph{inverses} if two properties are satisfied:
\begin{itemize}
  \item Whenever $x\in A=\dom(f)$, we have $g\bigl(f(x)\bigr)=x$.
  \item Whenever $x\in B=\dom(g)$, we have $f\bigl(g(x)\bigr)=x$.
\end{itemize}
A function $f:A\to B$ is \emph{invertible} if it has an inverse function $g:B\to A$. 
\end{defn}

\begin{example}{}{}
The function $f(x)=2x$ has inverse $f^{-1}(x)=\frac 12x$.
\end{example}

The example shows the naïve way of approaching inverses. As is common in calculus/algebra, we defined functions using formulæ. Thankfully, if we follow standard conventions regarding the implied domain and codomain, then $f:\R\to\R$ and there is no problem with our example! Unfortunately, things are rarely so straightforward for almost any interesting function.

% \begin{example}{}{}
% Consider the formulæ $f(x)=x^2+2$ and $g(x)=\sqrt{x-2}$.
% \begin{enumerate}
%   \item Suppose $\dom(f)=A=\{1,2,3,4\}$ and $\operatorname{codom}(f)=B=\{3,6,11,18\}$.\par
% \begin{minipage}[t]{0.7\linewidth}\vspace{-5pt}
% It is easily checked that $g$ is the \emph{inverse function} to $f$, as is easily checked:
% \[g\bigl(f(x)\bigr)=\sqrt{(x^2+2)-2}=x,\quad f\bigl(g(x)\bigr)=\bigl(\sqrt{x-2}\bigr)^2+2=x\]
% All we are doing is reversing inputs and outputs. More formally, the domain and range/codomain switch over:
% \end{minipage}\hfill
% \begin{minipage}[t]{0.29\linewidth}\vspace{0pt}
% \hfill\includegraphics{inverse-easy}
% \end{minipage}\smallbreak
% \begin{gather*}
% A=\dom(f)=\operatorname{range}(g)=\operatorname{codom}(g)\\
% B=\dom(g)=\operatorname{range}(f)=\operatorname{codom}(f)
% \end{gather*}
% \end{enumerate}
% \end{example}


\begin{example}{}{inversex2}
Let $f(x)=x^2-1$ and $g(x)=\sqrt{x+1}$; intuitively these are inverse functions; indeed\par
\begin{minipage}[t]{0.7\linewidth}\vspace{-12pt}
\[f\bigl(g(3)\bigr)=f(2)=3\quad\text{and}\quad g\bigl(f(3)\bigr)=g(8)=3\]
In both cases, the second function returns the input of the first. However, replacing positives with negatives causes both calculations to fail:
\begin{enumeratea}
  \item When working with real numbers, $f\bigl(g(-3)\bigr)=f(\sqrt{-2})$ is meaningless and doesn't recover $-3$. This is a very silly thing to write, since $-2$ is not in the implied domain $\dom(g)=[-1,\infty)$.
  \item $g\bigl(f(-3)\bigr)=g(8)=3\neq -3$. The problem here is more subtle, arising from our unthinking use of the implied domain $\dom(f)=\R$:\vspace{-12pt}
\end{enumeratea}
  \[\textcolor{Green}{-3}\in\dom(f)=\R\ \text{ but }\ \textcolor{orange}{-3}\not\in\operatorname{range}(g)=[0,\infty)\]
Both cases illustrate the problem in using a formula to define a function: imprecision regarding \emph{domains} and \emph{ranges.}
\end{minipage}\hfill\begin{minipage}[t]{0.29\linewidth}\vspace{-5pt}
\flushright\includegraphics[scale=0.9]{inverses-poly3}\\
\includegraphics[scale=0.95]{inverses-poly2}
\end{minipage}\medbreak
The simple fix, disallow negative inputs to $f$, is implemented by \emph{restricting its domain} to $[0,\infty)$. The graph of $g$ is unchanged, but we've deleted the \textcolor{Brown}{left side} of the graph of $f$. We now have the inverse relationships we desire:
\begin{itemize}\itemsep0pt
  \item If $x\in\dom(f)=[0,\infty)$, then $g\bigl(f(x)\bigr)=\sqrt{(x^2-1)+1}=x$
  \item If $x\in\dom(g)=[-1,\infty)$, then $f\bigl(g(x)\bigr)=\bigl(\sqrt{x+1}\bigr)^2-1=x$
\end{itemize}
\end{example}




There is a lot going on here, though hopefully the simple fix of restricting the domain feels intuitive and low-tech. Our goal is to formalize and generalize this discussion. In particular, given a function $f:A\to B$, we have two questions: is $f$ invertible and, if so, how do we compute its inverse?

\goodbreak


\boldinline{What makes a function invertible?}

To motivate, we return to the example and consider how the \emph{vertical line test} (Definition \ref{defn:function}) provides an alternative viewpoint. A function may be envisioned as first transporting an input \emph{vertically} from the $x$-axis until it hits the graph, then projecting it \emph{horizontally} to the $y$-axis. An inverse function should reverse this process:\par

\begin{minipage}[t]{0.72\linewidth}\vspace{-3pt}
\begin{quote}
Transport an element on the $y$-axis \emph{horizontally} until it hits the graph, then project vertically onto the $x$-axis.
\end{quote}
The reason this process fails in our example is that we have \emph{two choices} for how to transport 9 horizontally to the graph:
\begin{itemize}\itemsep2pt
  \item The \textcolor{Green}{left-hand choice} produces $-3$ but does not correspond to $g$.
  \item The \textcolor{orange}{right-hand choice} recovers the correct original value $\textcolor{orange}{3}=g(8)$.
\end{itemize}
\end{minipage}\hfill\begin{minipage}[t]{0.27\linewidth}\vspace{-8pt}
\flushright\includegraphics[scale=0.9]{inverses-poly1}
\end{minipage}\medbreak
The fix is to insist that $f$ passes a \emph{horizontal line test}: every such line should intersect the graph at most \emph{once}; this is exactly what we accomplish by restricting $\dom(f)=[0,\infty)$.\smallbreak
We get another useful observation for free. The inverse process (horizontal transport/vertical projection) is precisely what $g$ is doing once we switch the $x$ and $y$ axes: the \textcolor{blue}{graphs} of inverse functions are therefore related by \emph{reflection symmetry in the line $y=x$.}\smallbreak

Definition \ref{defn:inversefn} told us how to recognize an inverse function by its \emph{properties}. Our observation about reflections allows us to \emph{define} an inverse concretely for any function: when viewed as a set of ordered pairs $\bigl(a,f(a)\bigr)$, `reflection in the line $y=x$' simply means `reverse the order of each pair.'

\begin{defn}{}{}
Suppose $f:A\to B$ is a function. Its \emph{inverse} is the \emph{subset}
\[f^{-1}=\bigl\{(f(a),a):a\in A\bigr\}\subseteq B\times A\]
\end{defn}

The question of what makes a function invertible now becomes, `when is the \emph{set} $f^{-1}$ a function?' The answer to this is the vertical line test. Let's revisit our example.\par

\begin{example*}{\ref{ex:inversex2}, cont}{}
If $f(x)=x^2-1$ has domain $\R$, then its inverse is the set\par
\begin{minipage}[t]{0.65\linewidth}\vspace{-10pt}
\[f^{-1}=\bigl\{(y^2-1,y):y\in\R\bigr\}\]
obtained by reflecting the graph of $f$ in the line $y=x$. This is not a function because it fails the vertical line test in two ways:
\begin{enumerate}\itemsep2pt
  \item If \textcolor{magenta}{$x>-1$}, then the vertical line intersects the graph \emph{twice.}
  \item If \textcolor{cyan}{$x<-1$}, then the vertical line does not intersect the graph.
\end{enumerate}
Both issues have a simple fix:
\begin{enumerate}\itemsep2pt
  \item Restrict the \emph{domain} of $f$ to $A=[0,\infty)$.
  \item Restrict the \emph{codomain} of $f$ to $B=[-1,\infty)$.
\end{enumerate}
\end{minipage}\hfill\begin{minipage}[t]{0.34\linewidth}\vspace{-10pt}
\flushright\begin{tabular}{@{}c@{}}
\includegraphics{inverses-poly4}\\
Graph of $f^{-1}$
\end{tabular}
\end{minipage}\medbreak
The inverse of the \textcolor{blue}{restricted} function $f:A\to B$ is now a \emph{function} $f^{-1}:B\to A$, $f(x)=\sqrt{x+1}$:
\[f^{-1}=\bigl\{(y^2-1,y):y\in[0,\infty)\bigr\} =\bigl\{(x,\sqrt{x+1}),x\in[-1,\infty)\bigr\}\]
\end{example*}


\goodbreak

It's time to put this all together and justify what should be a familiar algorithm. The vertical line test for the reflected graph of $f^{-1}$ becomes a \emph{horizontal line test} for the original function $f$.

\begin{thm}{Horizontal Line Test}{}
Let $f:A\to B$ be a function. Then $f$ is invertible if and only if it satisfies two conditions:
\begin{enumerate}
  \item (1--1/one-to-one)\lstsp Distinct inputs have distinct outputs:\footnotemark
  \[f(x_1)=f(x_2)\implies x_1=x_2\]
  This is equivalent to every \textcolor{magenta}{horizontal line} intersecting the graph \textcolor{magenta}{at most once}.
  \item (Onto)\lstsp Every potential output is realized; otherwise said $B=\operatorname{range}(f)$:
  \[\text{Given $y\in B$, there is some $x\in A$ for which $y=f(x)$}\]
  This is equivalent to a \textcolor{cyan}{horizontal line} through $y\in B$ intersecting the graph \textcolor{cyan}{at least once}.
\end{enumerate}
\end{thm}

\footnotetext{This means the same as $x_1\neq x_2\implies f(x_1)\neq f(x_2)$ but is easier to calculate with!}

Our original version of $f(x)=x^2-1$ failed both conditions when $f:\R\to\R$, but passes both when $f:[0,\infty)\to[-1,\infty)$.\smallbreak

The horizontal line test becomes three-step algorithm for finding inverse functions, of which the first two steps are the horizontal line test:
\begin{enumerate}
  \item Check that $f$ is 1--1. If not, \emph{restrict its domain} $A=\dom(f)$.
	\item \emph{Define} $B=\operatorname{range}(f)$.
	\item Switch $x,y$ and solve $x=f(y)\rightsquigarrow y=f^{-1}(x)$ in terms of $x$.
\end{enumerate}

When $f:A\to B$ is invertible, the construction makes clear why we must have
\[\dom(f^{-1})=B=\operatorname{range}(f),\quad \operatorname{range}(f^{-1})=\dom(f)\]

\begin{example*}{\ref{ex:inversex2}, cont}{}
The first step in our algorithm involves some \emph{choice.} If we instead restrict the domain so that $\dom(f)=(-\infty,0]$, then the function is still 1--1:
\begin{align*}
f(x_1)=f(x_2)&\implies x_1^2-1=x_2^2-1 \implies (x_1-x_2)(x_1+x_2)=0 \implies x_1=\pm x_2\\
&\implies x_1=x_2
\end{align*}
since both $x_1,x_2\le 0$ they cannot have opposite signs!\par
\begin{minipage}[t]{0.7\linewidth}\vspace{-5pt}
In the second step we still have $B=\operatorname{range}(f)=[-1,\infty)$. The result is \emph{another inverse,} as we may compute via step 3:
\begin{align*}
x=f(y)=y^2-1&\implies y^2=x+1\\
&\implies f^{-1}(x)=y=-\sqrt{x+1}
\end{align*}
\end{minipage}\hfill\begin{minipage}[t]{0.29\linewidth}\vspace{-30pt}
\flushright\begin{tabular}{@{}c@{}}
\includegraphics{inverses-poly5}\\
\textcolor{Brown}{$f^{-1}(x)=-\sqrt{x+1}$}
\end{tabular}
\end{minipage}\medbreak
where we chose the \emph{negative} square root since $\operatorname{range}(f^{-1})=\dom(f)=(-\infty,0]$. This, of course, is the result of reflecting the  \textcolor{Brown}{brown} half of the graph of $f$ across the line $y=x$.
\end{example*}

\goodbreak

\fi

To finish we consider an algebraically tougher example and quickly list all the details.

\begin{example}{}{function-easy}
Let $y=f(x)=\frac 1{(2-x)^2}$. Its implied \textcolor{red}{\emph{domain}} consists of all real numbers except 2.
\begin{description}
	\begin{minipage}[t]{0.68\linewidth}\vspace{-7pt}
		\item The \textcolor{Green}{\emph{vertical line test}} is clearly visible on the graph: every vertical line $x=a$ (except $x=2$) intersects the graph exactly once.
		\item The \textcolor{Brown}{\emph{range}} is the interval $\R^+=(0,\infty)$ as can be seen by solving
		\[f(x)=y\iff \frac 1{2-x}=\pm \sqrt y\iff x=2\mp \frac 1{\sqrt y}\]
		Any positive output may be obtained via $y=f\bigl(2-\frac 1{\sqrt y}\bigr)$.\par
		The $\mp$ term shows that $f$ fails the \textcolor{orange}{\emph{horizontal line test}}: it isn't 1--1.
		\item There are two natural choices for an inverse:
		\begin{enumeratea}
		  \item Choose \textcolor{blue}{$\dom(f)=(2,\infty)$}, then $\pm\sqrt y=\frac 1{2-x}$ is \emph{negative} and we obtain the inverse function
		  \[g:(0,\infty)\to(2,\infty),\quad g(x)=2-\frac 1{\sqrt y}\] 
		  \item Choose \textcolor{Purple}{$\dom(f)=(-\infty,2)$}, then $\pm\sqrt y=\frac 1{2-x}$ is \emph{positive} and we obtain a second inverse function
		  \[h:(0,\infty)\to(\infty,2),\quad h(x)=2+\frac 1{\sqrt y}\] 
		\end{enumeratea}
	\end{minipage}\hfill\begin{minipage}[t]{0.31\linewidth}\vspace{-12pt}
	\flushright\includegraphics{functions-easyex1}\par
	\includegraphics{functions-easyex2}
	\end{minipage}
\end{description}
\end{example}




% \begin{example}{}{}
% Let $f(x)=x^3+8$.
% \begin{enumerate}
%   \item $f$ passes the horizontal line test for all $x$:
%   \[x_1^3+8=x_2^3+8\implies x_1=x_2\]
%   \item $x=y^3+8\implies y=f^{-1}(x)=\sqrt[3]{x-8}$.
% \end{enumerate}
% In this case the domain and range are both the whole real line.
% \end{example}

\begin{exercises}{}{}
\exstart If $\dom(f)=\R$, verify that $f(x)=x^3+8$ passes the horizontal line test and find its inverse.
\begin{enumerate}\setcounter{enumi}{1}
  \item Consider the function $f(x)=x^2+2x-3$. What do you have to do to the domain to make this invertible? Now find \emph{two inverses} of $f$.
  
  \item Sketch the graph of the following function with $\dom(f)=[0,3)$ and find \emph{three} distinct inverses:
  \[f(x)=\begin{cases}
  x&\text{if }0\le x<1\\
  x-1&\text{if }1\le x<2\\
  x-2&\text{if }2\le x<3
  \end{cases}\]
	

  \item Show that the following function $f:\R\to(\frac 32,\infty)$ is 1--1 and onto, sketch its graph and find $f^{-1}$. 
  \[f(x)=\begin{cases}
  3-\frac 12x&\text{if }x\le 2\\
  2-\frac 1x&\text{if }x>2
  \end{cases}\]
  
  \item (Hard)\lstsp Find the implied domain and range of $f(x)=\smash[b]{\frac{x+1}{1+\frac 1{x+1}}}$. Now find an interval on which $f$ is 1--1 and compute its inverse.
  
  \item An astute student observes that Definition \ref{defn:inversefn} only describes the \emph{properties} satisfied by \emph{an} inverse and asks why we keep referring to \emph{the} inverse. How would you respond?
  
\end{enumerate}
\end{exercises}

\iffalse
\clearpage



\subsection{Increasing \& Decreasing Functions and Convexity}



Quadratic polynomials are a good excuse to introduce several related concepts.

\begin{defn}{}{}
Let $f:I\to\R$ be a function defined on an interval $I$. We say that $f$ is:
\begin{enumerate}
  \item \emph{Increasing} if $x_0<x_1\implies f(x_0)\le f(x_1)$.
% 	\item \emph{Strictly increasing} if the second inequality is strict; $f(x_1)<f(x_2)$.
	\item \emph{Convex (concave up)} if $x_0\neq x_1\implies (1-t)f(x_0)+tf(x_1)\ge f\bigl((1-t)x_0+tx_1\bigr)$ for all $0<t<1$. Otherwise said, the line segment joining any two points on the graph lies \emph{above} the graph.
\end{enumerate}
Reverse the inequalities to obtain the notions of \emph{decreasing} and \emph{concave down.} The inequalities can also be made strict to obtain \emph{strictly increasing} and \emph{strictly convex,} etc.
\end{defn}

\begin{minipage}[t]{0.45\linewidth}\vspace{0pt}
The function in the picture is convex where its graph is blue. If $0<t<1$, the  value
\[x_t=(1-t)x_1+tx_2\]
lies strictly between $x_1$ and $x_2$. At this point the function lies below the corresponding point on the line joining $(x_1,f(x_1))$ and $(x_2,f(x_2))$:
\[f(x_t)<(1-t)f(x_1)+tf(x_2)\]
\end{minipage}\begin{minipage}[t]{0.55\linewidth}\vspace{0pt}
\flushright\includegraphics{poly-convex}
\end{minipage}\par

\begin{thm}{}{}
If $a>0$, then $f(x)=ax^2+bx+c$ is
\begin{enumerate}
  \item Increasing on the interval $[-\frac b{2a},\infty)$;
  \item Decreasing on the interval $(-\infty,-\frac b{2a}]$;
  \item Concave up on $\R$.
\end{enumerate}
The outcome is reversed if $a<0$.
\end{thm}

Instead of a formal argument, 
we consider the simplest example.

\begin{example}{}{}
$f(x)=x^2$ has apex $(0,0)$. Simply compare\par
\begin{minipage}[t]{0.65\linewidth}\vspace{0pt}
\[f(x_1)-f(x_0)=x_1^2-x_0^2 =(x_1-x_0)(x_1+x_0)\]
and consider two cases:
\begin{itemize}
  \item $0\le x_0<x_1\implies x_0+x_1>0\implies f(x_1)-f(x_0)>0$ so $f$ is \textcolor{blue}{strictly increasing}.
  \item $x_0<x_1\le 0\implies x_0+x_1<0\implies f(x_1)-f(x_0)<0$ so $f$ is \textcolor{Green}{strictly decreasing}.
\end{itemize}
Finally, for any $x_0,x_1\in\R$ and any $0<t<1$, a little algebra shows that
\end{minipage}\begin{minipage}[t]{0.35\linewidth}\vspace{-10pt}
\flushright\includegraphics{poly-quad3}
\end{minipage}

\begin{align*}
f\bigl((1-t)x_0+tx_1\bigr)-(1-t)f(x_0)+tf(x_1)&=\bigl(x_0+t(x_1-x_0)\bigr)^2-(1-t)x_0^2+tx_1^2\\
&=t(t-1)\bigl(x_0+x_1\bigr)^2<0
\end{align*}
whence $f$ is convex (concave up). An alternative argument is in the exercises.
\end{example}



\begin{exercises}{}{}
\end{exercises}
\fi